\documentclass[12pt,a4paper]{exam}
\usepackage[utf8]{inputenc}
\usepackage[brazil]{babel}
\usepackage{geometry}
\geometry{a4paper, margin=0.6cm}
\usepackage{graphicx} 

% Configuração da prova
\begin{document}

% --- Cabeçalho ---
\noindent
\textbf{Biologia I}  \hfill \textbf{FEGVEST} \\[0.4cm]
\textbf{Professor(a): Lucas Canto} \\[0.4cm]
\textbf{Aluno(a):}  \\[0.4cm]

\begin{center}
    {\Large \textbf{Lista de revisão 1:}} \\[0.2cm]
    \rule{\textwidth}{0.5pt}
\end{center}
\begin{center}
    {\Large \textbf{ características gerais das plantas, Briófitas e Pteridófitas}} \\[0.2cm]
    \rule{\textwidth}{0.5pt}
\end{center}

\vspace{0.5cm}

% --- Questões ---
\begin{questions}

\question As briófitas formam o grupo vegetal mais primitivo. Qual característica abaixo confirma essa ideia?\\
\begin{choices}
    \choice Caule de pequeno porte.
    \choice Ausência de vasos condutores.
    \choice Dependência da água para reprodução.
    \choice Vida em ambientes úmidos.\\\\
\end{choices}

\question No ciclo reprodutivo das Briófitas:\\
\begin{choices}
    \choice O gametofito é a fase dominante.
    \choice Os gametas masculinos e femininos são produzidos pelo mesmo indivíduo.
    \choice os anterozoides não dependem da água para chegarem à oosfera.
    \choice A meiose ocorre antes da produção dos gametas.\\\\
\end{choices}
\question Os principais representantes do grupo das briófitas são:\\
\begin{choices}
    \choice As musgos, os antóceros e as hepáticas.
    \choice As samambaias, os musgos e as hepáticas.
    \choice As gramíneas, os os antóceros e as samambaias.
    \choice Os musgos, as gramíneas e os antóceros.\\\\
\end{choices}
\question As pteridofitas são...\\
\begin{choices}
    \choice Plantas com  vasos condutores.
    \choice Plantas sem vasos condutores.
    \choice Plantas com rizoide.
    \choice Plantas sem dependencia de agua.\\\\
\end{choices}

\question (Fuvest-2021)Considere três espécies de plantas (X, Y e Z) e suas características:\\\\
- A planta X não possui flores, mas é polinizada pelo vento. Além disso, não possui frutos, mas suas sementes são dispersas por aves.

- A planta Y não possui flores, nem sementes, nem frutos.

- A planta Z possui flores e é polinizada por aves. Além disso, possui frutos e suas sementes são dispersas por aves.\\\\
A que grupos pertencem as plantas X, Y e Z, respectivamente?\\
\begin{choices}
    \choice Pteridófitas, angiospermas e gimnospermas.
    \choice Gimnospermas, pteridófitas e angiospermas.
    \choice 
Pteridófitas, gimnospermas e angiospermas.
    \choice Angiospermas, gimnospermas e pteridófitas.
    \choice Gimnospermas, angiospermas e pteridófitas.\\
\end{choices}

\question (UNICAMP-2016)De acordo com o cladograma a seguir, é correto afirmar que:
\begin{center}
    \includegraphics[width=0.5\textwidth]{cladograma.png} 
\end{center}
\begin{choices}
    \choice A é Briófita, B é Pteridófita e C é Espermatófita
    \choice C é Espermatófita, D é traqueófita e E é Angiosperma.
    \choice C possui sementes, D é Espermatófita e E é Angiosperma
    \choice B é Briófita, D é traqueófita e E possui sementes.\\
\end{choices}

\question (UECE-CEV-2021)Considerando as características das briófitas e pteridófitas, numere os parênteses abaixo de acordo com a seguinte indicação:
1. briófitas; 2. pteridófitas.\\\\
( ) Seu tamanho está associado à ausência de vasos para a condução dos nutrientes, os quais são transportados de célula a célula por todo o vegetal.\\ ( ) Samambaias, avencas, xaxins e cavalinhas são alguns dos seus representantes mais conhecidos.\\ ( ) Os musgos e as hepáticas são seus principais representantes.\\ ( ) Foram os primeiros vegetais a apresentar um sistema de vasos para conduzir nutrientes.
A sequência correta, de cima para baixo, é:\\
\begin{choices}
    \choice 1, 1, 2, 2.
    \choice 1, 2, 1, 2.
    \choice 2, 1, 2, 1.
    \choice 2, 2, 1, 1.\\
\end{choices}


\end{questions}

\end{document}